% Copyright 2020 by Junwei Wang <i.junwei.wang@gmail.com>
%
% This file may be distributed and/or modified under the
% conditions of the LaTeX Project Public License, either version 1.3c
% of this license or (at your option) any later version.
% The latest version of this license is in
%   http://www.latex-project.org/lppl.txt

\documentclass[compress]{beamer}

\usepackage[english]{babel}
\usepackage{metalogo}
\usepackage{listings}
\usepackage{fontspec}
\usepackage{tikz}
% \usepackage{andika}

\usetheme{Nord}
% \usetheme[style=light]{Nord}

% \setmainfont{Yanone Kaffeesatz}
% \setsansfont{Andika New Basic}
% \setmonofont{DejaVu Sans Mono}


\setmainfont{YanoneKaffeesatz-Regular}
\setsansfont{YanoneKaffeesatz-Regular}
\setmonofont{DejaVuSansMono}

\AtBeginSection[]
{
  \begin{frame}[c,noframenumbering,plain]
    \tableofcontents[sectionstyle=show/hide,subsectionstyle=show/show/hide]
  \end{frame}
}

\AtBeginSubsection[]
{
  \begin{frame}[c,noframenumbering,plain]
    \tableofcontents[sectionstyle=show/hide,subsectionstyle=show/shaded/hide]
  \end{frame}
}

\title{Beamer Theme ``Nord''}
\subtitle{A simple beamer theme that uses ``Nord'' color scheme}
\author{Junwei Wang}
\institute{CryptoExperts}
\date{\today}

\begin{document}

\begin{frame}[plain,noframenumbering]
  \maketitle
\end{frame}


\section{Appearance}


\begin{frame}[fragile]
  \frametitle{Usage}
  Simply include the following code in your preamble:

  \begin{lstlisting}[basicstyle = \ttfamily\small]
\usetheme{Nord}
  \end{lstlisting}

  \bigskip

  By default, the appearance is in dark theme, however you can actively choose a either a light or a
  dark theme.

  \begin{lstlisting}[basicstyle = \ttfamily\small]
\usetheme[style=light]{Nord}
\usetheme[style=dark]{Nord}
  \end{lstlisting}

\end{frame}

\subsection{Colors}

\begin{frame}{Defined Colors}{This is a subtitle}
  \begin{description}[Snow Storm]
  \item[Polar Night]
    \textcolor{NordDarkBlack}{NordDarkBlack} \quad \textcolor{NordBlack}{NordBlack}\\
    \textcolor{NordMediumBlack}{NordMediumBlack} \quad \textcolor{NordBrightBlack}{NordBrightBlack}
  \item[Snow Storm]
    \textcolor{NordWhite}{NordWhite} \quad \textcolor{NordBrighterWhite}{NordBrightestWhite}\\
    \textcolor{NordBrightestWhite}{NordBrightestWhite}
  \item[Forest]
    \textcolor{NordCyan}{NordCyan} \quad \textcolor{NordBrightCyan}{NordBrightCyan}\\
    \textcolor{NordBlue}{NordBlue} \quad \textcolor{NordBrightBlue}{NordBrightBlue}
  \item[Aurora]
    \textcolor{NordRed}{NordRed} \quad \textcolor{NordOrange}{NordOrange} \\
    \textcolor{NordYellow}{NordYellow} \quad \textcolor{NordGreen}{NordGreen} \\
    \textcolor{NordMagenta}{NordMagenta}
  \end{description}
\end{frame}

\subsection{Fonts}

\begin{frame}[fragile]{Recommended Free Fonts}
  \begin{description}[Selected Fonts]
  \item[Selected Fonts] recommended for this theme\\
    \begin{lstlisting}[basicstyle = \ttfamily\small]
\setmainfont{Yanone Kaffeesatz}
\setsansfont{Andika New Basic}
\setmonofont{DejaVu Sans Mono}
    \end{lstlisting}
  \item[Download] {\small \url{https://www.fontsquirrel.com/}}
  \item[Install Fonts] {\small \url{https://www.google.com/get/noto/help/install/}}
  \item[Compilation] compile with \XeLaTeX~to use system-wide fonts
  \end{description}

\end{frame}


\subsection{Blocks}

\begin{frame}
  \frametitle{Blocks}
  \begin{block}{This is a Block}
    \[
      a^2 + b^2 = c^2
    \]
  \end{block}
  \begin{exampleblock}{This is an Example Block}
    \[
      E = m \cdot c^{2}
    \]
  \end{exampleblock}
  \begin{alertblock}{This is an Alert Block}
    \[
      e^{i\pi} + 1 = 0
    \]
  \end{alertblock}

  \centering
  \begin{minipage}{1.0\linewidth}
    \begin{block}{Horizontally-Aligned Block}
      \[
        \log xy = \log x + \log y
      \]
    \end{block}
  \end{minipage}
\end{frame}

\subsection{Items}

\begin{frame}{Items}
  Itemize
  \begin{itemize}
    \item item 1
    \item item 2
  \end{itemize}

  \bigskip

  Enumerate
  \begin{enumerate}
    \item item 1
    \item item 2
  \end{enumerate}
\end{frame}

\subsection{Figures}

\begin{frame}{Figures}
  \begin{figure}
    \centering
    \begin{tikzpicture}
      \draw [help lines,NordMagenta,very thick] (0,0) grid (5,4);
    \end{tikzpicture}
    \caption{Credits to Ti\textit{k}Z}
  \end{figure}
\end{frame}

\end{document}

%%% Local Variables:
%%% mode: latex
%%% TeX-master: t
%%% TeX-engine: xetex
%%% End:
